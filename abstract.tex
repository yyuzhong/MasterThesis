%%%%%%%%%%%%%%%%%%%%%%%%%%%%%%%%%%%%%%%%%%%%%%%%%%%
%
%  New template code for TAMU Theses and Dissertations starting Fall 2012.  
%  For more info about this template or the 
%  TAMU LaTeX User's Group, see http://www.howdy.me/.
%
%  Author: Wendy Lynn Turner 
%	 Version 1.0 
%  Last updated 8/5/2012
%
%%%%%%%%%%%%%%%%%%%%%%%%%%%%%%%%%%%%%%%%%%%%%%%%%%%
%%%%%%%%%%%%%%%%%%%%%%%%%%%%%%%%%%%%%%%%%%%%%%%%%%%%%%%%%%%%%%%%%%%%%
%%                           ABSTRACT 
%%%%%%%%%%%%%%%%%%%%%%%%%%%%%%%%%%%%%%%%%%%%%%%%%%%%%%%%%%%%%%%%%%%%%

\chapter*{ABSTRACT}
\addcontentsline{toc}{chapter}{ABSTRACT} % Needs to be set to part, so the TOC doesnt add 'CHAPTER ' prefix in the TOC.

\pagestyle{plain} % No headers, just page numbers
\pagenumbering{roman} % Roman numerals
\setcounter{page}{2}

\indent Seismic data analytics that processes and interprets multi-dimensional seismic volumes plays a key role in oil\&gas exploration. Developing any new seismic algorithms currently requires a team of geophysicists, IT developers and High Performance Computing (HPC) experts working together; designing algorithms in high-level programming interfaces, verifying these algorithms with small sample data, and then translating into low-level but efficient parallel MPI codes with optimization to handle actual big data. The whole process is time consuming, inefficient, and sometimes even lead to inconsistent results between experiment data and actual data. Apache Spark is a new big data analytics platform that support more than map/reduce parallel execution mode with good scalability and fault tolerance. In this thesis, a template-based seismic data analytics cloud platform was proposed and implemented with emphasis on both performance and productivity. With this platform, geophysicists and data scientists could create their algorithms and directly experiment them with actual data using our defined templates, so as to improve both performance and productivity. Our platform will handle the data distribution, code generation and execute the program in parallel. We will use several user cases to demonstrate the performance and productivity of the platform on our computer cluster.  


 

\pagebreak{}
